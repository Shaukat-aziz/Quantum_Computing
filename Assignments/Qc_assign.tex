
\documentclass{article}
\usepackage{amsmath,amssymb}
\usepackage{geometry}
\geometry{
  a4paper,
  margin=1in,
}

\begin{document}

\title{Assignment-3 Solutions}
\author{Shaukat aziz}
\date{QT 207}
\maketitle

\section*{II. Quantum Fourier Transform (QFT) and Unitary Operator}

\subsection*{Exercise 2: Find an operator $\hat{F}$ which transforms a state into its DFT and show that it is unitary.}

\subsubsection*{Defining the QFT Operator $\hat{F}$}

Consider an $N$-dimensional Hilbert space with computational basis
\[
\{|j\rangle\}_{j=0}^{N-1}.
\]
The quantum Fourier transform (QFT) operator $\hat{F}$ is defined by its action on the basis states as
\begin{equation}
  \hat{F}\,|j\rangle
  = \frac{1}{\sqrt{N}}\sum_{k=0}^{N-1} e^{2\pi i jk/N}\,|k\rangle,
  \qquad j = 0,1,\dots,N-1.
\end{equation}

In matrix form,
\[
\hat{F} = \sum_{k=0}^{N-1} \sum_{j=0}^{N-1}
\langle k|\hat{F}|j\rangle\,|k\rangle \langle j| ,
\]
with matrix elements
\begin{equation}
  \langle k|\hat{F}|j\rangle
  = \frac{1}{\sqrt{N}} e^{\frac{2\pi i}{N} jk}.
\end{equation}
Therefore,
\begin{equation}
  \hat{F}
  = \frac{1}{\sqrt{N}}\sum_{k=0}^{N-1} \sum_{j=0}^{N-1}
  e^{\frac{2\pi i}{N} jk}\,|k\rangle\langle j|.
\end{equation}

If a general state is
\[
|\psi\rangle = \sum_{j=0}^{N-1} x_j |j\rangle,
\]
then
\[
\hat{F}|\psi\rangle
= \sum_{k=0}^{N-1} \left(
  \frac{1}{\sqrt{N}} \sum_{j=0}^{N-1} e^{2\pi i jk/N} x_j
\right) |k\rangle,
\]
so the amplitudes are exactly the DFT of $\{x_j\}$.

\subsubsection*{Showing that $\hat{F}$ is Unitary}

An operator is unitary if
\[
\hat{F}\hat{F}^\dagger = \hat{I}
\quad\text{and}\quad
\hat{F}^\dagger\hat{F} = \hat{I}.
\]

The matrix elements of $\hat{F}^\dagger$ are
\[
\langle j|\hat{F}^\dagger|k\rangle
= \overline{\langle k|\hat{F}|j\rangle}
= \frac{1}{\sqrt{N}} e^{-2\pi i jk/N}.
\]

Consider the matrix element of $\hat{F}\hat{F}^\dagger$:
\begin{align*}
(\hat{F}\hat{F}^\dagger)_{kk'}
&= \langle k|\hat{F}\hat{F}^\dagger|k'\rangle \\
&= \sum_{j=0}^{N-1} \langle k|\hat{F}|j\rangle \langle j|\hat{F}^\dagger|k'\rangle \\
&= \sum_{j=0}^{N-1}
\left(\frac{1}{\sqrt{N}} e^{2\pi i jk/N}\right)
\left(\frac{1}{\sqrt{N}} e^{-2\pi i jk'/N}\right) \\
&= \frac{1}{N} \sum_{j=0}^{N-1}
e^{2\pi i j(k-k')/N}.
\end{align*}
The sum over roots of unity gives
\[
\sum_{j=0}^{N-1}
e^{2\pi i j(k-k')/N}
=
\begin{cases}
N, & k = k',\\[2pt]
0, & k \neq k'.
\end{cases}
\]
Hence
\[
(\hat{F}\hat{F}^\dagger)_{kk'}
= \delta_{kk'}, \qquad
\hat{F}\hat{F}^\dagger = \hat{I}.
\]

By the same calculation (or by the fact that $\hat{F}$ is a square matrix with orthonormal columns), one also finds
\[
\hat{F}^\dagger \hat{F} = \hat{I}.
\]
Therefore $\hat{F}$ is unitary.

\bigskip
\hrule
\bigskip

\section*{III. Discrete Fourier Transform (DFT) on Periodic States}

\subsection*{Exercise 3: Performing DFT on the periodic state $|\phi_{r,b}\rangle \to |\tilde{\phi}\rangle$, obtain the resulting state.}

The periodic state is
\begin{equation}
  |\phi_{r,b}\rangle
  = \frac{1}{\sqrt{m}} \sum_{z=0}^{m-1} |zr + b\rangle,
\end{equation}
where $r$ is the period, $b$ is an offset, and $m$ is the number of repetitions.

Let the Hilbert space dimension be $N$, and assume $N = mr$ so that $m = N/r$ is an integer. The QFT (DFT) modulo $N$ acts as
\begin{equation}
  \hat{F}_N |x\rangle =
  \frac{1}{\sqrt{N}} \sum_{k=0}^{N-1} e^{2\pi i xk/N}\,|k\rangle.
\end{equation}

Apply $\hat{F}_N$ to $|\phi_{r,b}\rangle$:
\begin{align*}
\hat{F}_N |\phi_{r,b}\rangle
&= \frac{1}{\sqrt{m}} \sum_{z=0}^{m-1} \hat{F}_N |zr + b\rangle \\
&= \frac{1}{\sqrt{m}} \sum_{z=0}^{m-1}
\left(
  \frac{1}{\sqrt{N}} \sum_{k=0}^{N-1} e^{2\pi i (zr + b)k/N} |k\rangle
\right) \\
&= \frac{1}{\sqrt{mN}} \sum_{k=0}^{N-1} e^{2\pi i bk/N}
\left(
  \sum_{z=0}^{m-1} e^{2\pi i z r k / N}
\right) |k\rangle.
\end{align*}

Define the inner sum
\[
S(k) = \sum_{z=0}^{m-1} e^{2\pi i z r k / N}.
\]
Using $N = mr$, this becomes
\[
S(k) = \sum_{z=0}^{m-1} e^{2\pi i z k / m}.
\]

This is a geometric series with common ratio $e^{2\pi i k/m}$. Hence
\[
S(k) =
\begin{cases}
m, & k \equiv 0 \pmod{m},\\
0, & k \not\equiv 0 \pmod{m}.
\end{cases}
\]

Thus only $k$ that are multiples of $m$ survive. Write $k = t m$ with $t = 0,1,\dots,r-1$:
\begin{align*}
\hat{F}_N |\phi_{r,b}\rangle
&= \frac{1}{\sqrt{mN}} \sum_{t=0}^{r-1}
e^{2\pi i b (tm)/N} \, m \, |tm\rangle \\
&= \frac{\sqrt{m}}{\sqrt{N}}
\sum_{t=0}^{r-1} e^{2\pi i b t / r} \, |tm\rangle.
\end{align*}
Since $m = N/r$,
\[
\frac{\sqrt{m}}{\sqrt{N}} = \sqrt{\frac{m}{mr}} = \frac{1}{\sqrt{r}},
\]
and $tm = tN/r$. Therefore the Fourier-transformed state is
\begin{equation}
  |\tilde{\phi}\rangle
  = \hat{F}_N |\phi_{r,b}\rangle
  = \frac{1}{\sqrt{r}}
    \sum_{t=0}^{r-1} e^{2\pi i b t / r} \left|\frac{tN}{r}\right\rangle.
\end{equation}

\bigskip
\hrule
\bigskip

\section*{IV. Relation Between Hadamard and QFT}

\subsection*{Exercise 4: Show that $\displaystyle \frac{1}{\sqrt{q}}\sum_{y=0}^{q-1}|y\rangle$ can be obtained by applying Hadamard and also by applying Fourier transform. What does this tell you about the relation between Hadamard and Fourier transform?}

Let $q = 2^k$. The first register has $k$ qubits, initialized to $|0\rangle^{\otimes k}$.

\subsubsection*{Using the Hadamard Gate}

For a single qubit,
\[
\hat{H}|0\rangle = \frac{1}{\sqrt{2}}(|0\rangle + |1\rangle).
\]
Therefore,
\begin{align*}
\hat{H}^{\otimes k} |0\rangle^{\otimes k}
&= \bigotimes_{j=1}^{k} \hat{H}_j |0\rangle_j \\
&= \bigotimes_{j=1}^{k} \frac{1}{\sqrt{2}}(|0\rangle_j + |1\rangle_j) \\
&= \frac{1}{\sqrt{2^k}}
   \sum_{y_1,\dots,y_k \in \{0,1\}}
   |y_1 y_2 \dots y_k\rangle.
\end{align*}
Identifying $y = y_1 y_2 \dots y_k$ as the binary representation of $y \in \{0,\dots,q-1\}$, we get
\begin{equation}
  \hat{H}^{\otimes k} |0\rangle^{\otimes k}
  = \frac{1}{\sqrt{q}} \sum_{y=0}^{q-1} |y\rangle.
\end{equation}

\subsubsection*{Using the QFT}

The QFT on $q = 2^k$ basis states is defined as
\[
\hat{F}_q |j\rangle
= \frac{1}{\sqrt{q}}\sum_{y=0}^{q-1}
e^{2\pi i j y / q}|y\rangle.
\]
Apply it to $|0\rangle$:
\begin{equation}
  \hat{F}_q |0\rangle
  = \frac{1}{\sqrt{q}}\sum_{y=0}^{q-1}
  e^{2\pi i \cdot 0 \cdot y / q}|y\rangle
  = \frac{1}{\sqrt{q}}\sum_{y=0}^{q-1}|y\rangle.
\end{equation}

\subsubsection*{Conclusion}

Both operations produce the same equal superposition from the initial state $|0\rangle^{\otimes k}$:
\[
\hat{H}^{\otimes k}|0\rangle^{\otimes k}
= \hat{F}_q |0\rangle.
\]
This shows:

\begin{itemize}
  \item For $N=2$, the single-qubit Hadamard gate is exactly the QFT on $\mathbb{Z}_2$.
  \item For $N=2^k$, $\hat{H}^{\otimes k}$ and $\hat{F}_q$ coincide on $|0\rangle^{\otimes k}$ and are closely related. The QFT over $\mathbb{Z}_{2^k}$ can be built from Hadamard gates plus controlled phase rotations.
\end{itemize}

So the Hadamard transform is a special (very simple) case of a quantum Fourier transform.

\bigskip
\hrule
\bigskip

\section*{V. Destructive Interference in QFT}

\subsection*{Exercise 5: The sum changed from $q/r$ terms to $r$ terms during QFT. This was a result of destructive interference in the QFT on the state. Show how it happened.}

From Shor's order-finding procedure, after measuring the second register we obtain in the first register the state
\begin{equation}
  |\phi_{a_0}\rangle
  = \frac{1}{\sqrt{q/r}}
    \sum_{z=0}^{q/r - 1} |zr + a_0\rangle,
\end{equation}
where $q = mr$ and hence $q/r = m$.

Apply the QFT modulo $q$:
\[
\hat{F}_q |x\rangle
= \frac{1}{\sqrt{q}}\sum_{k=0}^{q-1}
e^{2\pi i xk/q} |k\rangle.
\]

Then
\begin{align*}
\hat{F}_q |\phi_{a_0}\rangle
&= \frac{1}{\sqrt{m}}
\sum_{z=0}^{m-1} \hat{F}_q |zr + a_0\rangle \\
&= \frac{1}{\sqrt{m}}
\sum_{z=0}^{m-1}
\left(
  \frac{1}{\sqrt{q}} \sum_{k=0}^{q-1}
  e^{2\pi i (zr + a_0)k/q}|k\rangle
\right) \\
&= \frac{1}{\sqrt{mq}} \sum_{k=0}^{q-1}
e^{2\pi i a_0 k / q}
\left(
  \sum_{z=0}^{m-1} e^{2\pi i z r k / q}
\right) |k\rangle.
\end{align*}

Again define
\[
S(k) = \sum_{z=0}^{m-1} e^{2\pi i z r k / q}.
\]
Using $q = mr$, we get
\[
S(k) = \sum_{z=0}^{m-1} e^{2\pi i z k / m}.
\]

This is the same geometric series as before:
\[
S(k) =
\begin{cases}
m, & k \equiv 0 \pmod{m},\\[2pt]
0, & k \not\equiv 0 \pmod{m}.
\end{cases}
\]

\paragraph{Destructive interference.}
If $k$ is not a multiple of $m = q/r$, then $e^{2\pi i k/m} \neq 1$, and the terms
\[
e^{2\pi i z k/m},\quad z=0,\dots,m-1
\]
sum to zero. That is,
\[
S(k)=0 \quad \Rightarrow \quad \text{amplitude of }|k\rangle = 0.
\]
All such components are removed by destructive interference.

\paragraph{Constructive interference.}
If $k$ \emph{is} a multiple of $m$, say $k = t m$ with $t=0,1,\dots,r-1$, then
\[
e^{2\pi i z k/m} = e^{2\pi i z t} = 1
\]
for all $z$, so
\[
S(k) = \sum_{z=0}^{m-1} 1 = m.
\]
Thus for $k = t m$,
\begin{align*}
\hat{F}_q |\phi_{a_0}\rangle
&= \frac{1}{\sqrt{mq}} \sum_{t=0}^{r-1}
e^{2\pi i a_0 (tm) / q} \, m \, |tm\rangle \\
&= \frac{\sqrt{m}}{\sqrt{q}}
\sum_{t=0}^{r-1}
e^{2\pi i a_0 t / r} \, |tm\rangle.
\end{align*}
Since $m = q/r$, we have $\sqrt{m/q} = 1/\sqrt{r}$, and $tm = t q/r$, so
\begin{equation}
  \hat{F}_q |\phi_{a_0}\rangle
  = \frac{1}{\sqrt{r}} \sum_{t=0}^{r-1}
    e^{2\pi i a_0 t / r} \left|\frac{tq}{r}\right\rangle.
\end{equation}

Originally, $|\phi_{a_0}\rangle$ was a superposition of $q/r = m$ basis states. After the QFT, only $r$ basis states (those with $k = t q/r$) have non-zero amplitude. The reduction from $q/r$ terms to $r$ terms is exactly due to the destructive interference of all other components in the geometric sum $S(k)$.

\end{document}
```
