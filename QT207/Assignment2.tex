\documentclass[oneside,openany]{book}
% Core Packages
\usepackage{amsmath,amsfonts,amsthm,amssymb,mathtools}
\usepackage{geometry}
\usepackage[most,many,breakable]{tcolorbox}
\usepackage{xcolor}
% \usepackage{tikz} % No longer needed for this tcolorbox style
\usepackage{physics} % For \ket, \bra, \expval

\geometry{a4paper, margin=1in}

% Use book class but keep section numbers independent of chapter (no 0.1, 1.1 etc.)
\renewcommand\thesection{\arabic{section}}

%%%%%%%%%%%%%%%%%%%%%%%%%%%%%%
% SELF MADE COLORS (from your preamble)
%%%%%%%%%%%%%%%%%%%%%%%%%%%%%%
\definecolor{myg}{RGB}{56, 140, 70} % Green for Solution
\definecolor{myb}{RGB}{45, 111, 177} % Blue for Problem

%%%%%%%%%%%%%%%%%%%%%%%%%%%%
% TCOLORBOX SETUPS (Inspired Style)
%%%%%%%%%%%%%%%%%%%%%%%%%%%%
\tcbuselibrary{theorems,skins,hooks} % Added library for the new style

% 1. PROBLEM BOX (Based on your inspiration)
\newtcbtheorem{Exercise}{Exercise}{
    enhanced,
    breakable,
    colback = myb!10,              % Light blue background
    frame hidden,
    boxrule = 0sp,
    borderline west = {2pt}{0pt}{myb}, % Thick blue left border
    sharp corners,
    detach title,
    before upper = \tcbtitle\par\smallskip, % Title on top
    coltitle = myb,                        % Title color blue
    fonttitle = \bfseries\sffamily,        % Sans-serif bold title
    separator sign none,
}{pr}
\newcommand{\pr}[2]{\begin{Exercise*}[theorem name={}]{#1}{}#2\end{Exercise*}}

% 2. SOLUTION BOX (Based on your inspiration)
\newtcolorbox{solution}{
    enhanced,
    breakable,
    colback = myg!10,               % Light green background
    frame hidden,
    boxrule = 0sp,
    borderline west = {2pt}{0pt}{myg}, % Thick green left border
    sharp corners,
    detach title,
    title = Solution,      % Fixed title
    before upper = \tcbtitle\par\smallskip, % Title on top
    coltitle = myg,                         % Title color green
    fonttitle = \bfseries\sffamily,         % Sans-serif bold title
    separator sign none,
}
\newcommand{\sol}[1]{\begin{solution}#1\end{solution}} % Solution shortcut


\title{Assignment 2 - QT 207}
\author{Z SHAUKAT AZIZ (22171)}
\date{20th October 2025}


\begin{document}

\maketitle

\section{Quantum States and Entanglement}
\setcounter{section}{3} % Set section counter to 3 for Exercise 3.x.x
\subsection*{ 3.3.1: Entanglement of the Bell State}

\pr{ 3.3.1}{
Consider the 2-qubit state $|\psi\rangle=\frac{1}{\sqrt{2}}|0\rangle|0\rangle+\frac{1}{\sqrt{2}}|1\rangle|1\rangle$. Show that this state is \textbf{entangled} by proving that there are no possible values $\alpha_0, \alpha_1, \beta_0, \beta_1$ such that:
$$|\psi\rangle=(\alpha_{0}|0\rangle+\alpha_{1}|1\rangle)(\beta_{0}|0\rangle+\beta_{1}|1\rangle).$$
}

\sol{
For $|\psi\rangle$ to be a product state, we must be able to equate the coefficients of the computational basis states in the expansion of the product state to the given state:
$$ (\alpha_{0}\beta_{0})|00\rangle + (\alpha_{0}\beta_{1})|01\rangle + (\alpha_{1}\beta_{0})|10\rangle + (\alpha_{1}\beta_{1})|11\rangle = \frac{1}{\sqrt{2}}|00\rangle + 0|01\rangle + 0|10\rangle + \frac{1}{\sqrt{2}}|11\rangle $$
Equating coefficients yields a system of four equations:
\begin{enumerate}
    \item $\alpha_{0}\beta_{0} = \frac{1}{\sqrt{2}}$
    \item $\alpha_{0}\beta_{1} = 0$
    \item $\alpha_{1}\beta_{0} = 0$
    \item $\alpha_{1}\beta_{1} = \frac{1}{\sqrt{2}}$
\end{enumerate}
\begin{itemize}
    \item From (1) and (4), since the Right-Hand Side is non-zero, all four coefficients $\alpha_0, \alpha_1, \beta_0, \beta_1$ must be non-zero individually.
    \item From (2): $\alpha_{0}\beta_{1} = 0$. Since $\alpha_0 \ne 0$ (from Eq. 1), this requires \textbf{$\beta_{1} = 0$}.
    \item From (3): $\alpha_{1}\beta_{0} = 0$. Since $\beta_0 \ne 0$ (from Eq. 1), this requires \textbf{$\alpha_{1} = 0$}.
\end{itemize}
Substituting $\alpha_1=0$ and $\beta_1=0$ into (4) gives:
$$ \alpha_{1}\beta_{1} = (0)(0) = 0 $$
The system of equations is incosistent and does not satisfy all four equations simultaneously. Thus, there are no values of $\alpha_0, \alpha_1, \beta_0, \beta_1$ that satisfy the equations. $\implies$ the state $|\psi\rangle$ is \textbf{entangled}.
i.e cannot be written as a product state.
}

\subsection*{Exercise 3.4.1: Properties of Projectors and State Decomposition}

\pr{ 3.4.1 (a)}{
Prove that if the operators $P_{i}$ satisfy $P_{i}^{\dagger}=P_{i}$ (Hermitian) and $P_{i}^{2}=P_{i}$ (Projector), then $P_{i}P_{j}=0$ for all $i\ne j$. (Assume $\sum_k P_k = I$).
}

\sol{
We assume the set of projectors $\{P_k\}$ forms a **complete** measurement, meaning the identity operator $I$ can be decomposed as $I = \sum_{k} P_{k}$.
\begin{enumerate}
    \item Start with the completeness relation:
    $$ \sum_{k} P_{k} = I $$
    \item Multiply by $P_i$ on the left:
    $$ P_i \left( \sum_{k} P_{k} \right) = P_i I \implies \sum_{k} P_i P_{k} = P_i $$
    \item Separate the term where $k=i$:
    $$ P_i P_i + \sum_{k \ne i} P_i P_{k} = P_i $$
    \item Since $P_i$ is a projector, $P_i^2 = P_i$:
    $$ P_i + \sum_{k \ne i} P_i P_{k} = P_i $$
    \item Subtract $P_i$ from both sides:
    $$ \sum_{k \ne i} P_i P_{k} = 0 $$
    Since $P_i$ and $P_j$ are orthogonal projectors, the operator $\sum_{k \ne i} P_i P_k$ is a sum of positive semi-definite operators. Since this sum is the zero operator, each term in the sum must be the zero operator.
\end{enumerate}
Thus, $\mathbf{P_i P_j = 0}$ for $i \ne j$.
}

\pr{ 3.4.1 (b)}{
Prove that any pure state $|\psi\rangle$ can be decomposed as $|\psi\rangle=\sum_{i}\alpha_{i}|\psi_{i}\rangle$ where $\alpha_{i}=\sqrt{p(i)}$, $p(i)=\langle\psi|P_{i}|\psi\rangle$, and $|\psi_{i}\rangle=\frac{P_{i}|\psi\rangle}{\sqrt{p(i)}}$. Also prove that $\langle\psi_{i}|\psi_{j}\rangle=\delta_{i,j}$.
}

\sol{
\textbf{Decomposition:}
Using the completeness relation $I = \sum_i P_i$:
$$ |\psi\rangle = I |\psi\rangle = \sum_{i} P_{i} |\psi\rangle $$
We define the normalized state $|\psi_i\rangle$ by first considering the unnormalized state $|\phi_i\rangle = P_i |\psi\rangle$.
The squared norm of $|\phi_i\rangle$ is:
$$ \expval{\phi_i | \phi_i} = \langle\psi|P_i^\dagger P_i|\psi\rangle = \langle\psi|P_i^2|\psi\rangle = \langle\psi|P_i|\psi\rangle = p(i) $$
We define the coefficients $\alpha_i = \sqrt{p(i)}$ and the normalized states $|\psi_i\rangle = \frac{P_i |\psi\rangle}{\sqrt{p(i)}}$.
Substituting back into the decomposition:
$$ P_i |\psi\rangle = \sqrt{p(i)} \frac{P_i |\psi\rangle}{\sqrt{p(i)}} = \alpha_i |\psi_i\rangle $$
Therefore, $|\psi\rangle = \sum_{i} P_{i} |\psi\rangle = \sum_{i} \alpha_{i}|\psi_{i}\rangle$.

\textbf{Orthogonality:}
We calculate the inner product $\langle\psi_{i}|\psi_{j}\rangle$:
$$ \langle\psi_{i}|\psi_{j}\rangle = \left( \frac{\langle\psi|P_{i}^\dagger}{\sqrt{p(i)}} \right) \left( \frac{P_{j}|\psi\rangle}{\sqrt{p(j)}} \right) = \frac{1}{\sqrt{p(i)p(j)}} \langle\psi|P_{i}P_{j}|\psi\rangle $$
\begin{itemize}
    \item \textbf{If $i \ne j$}: $P_i P_j = 0$ (from part a).
    $$ \langle\psi_{i}|\psi_{j}\rangle = \frac{1}{\sqrt{p(i)p(j)}} \langle\psi|0|\psi\rangle = 0 $$
    \item \textbf{If $i = j$}: $P_i P_i = P_i$ and $\langle\psi|P_i|\psi\rangle = p(i)$.
    $$ \langle\psi_{i}|\psi_{i}\rangle = \frac{1}{p(i)} \langle\psi|P_{i}|\psi\rangle = \frac{p(i)}{p(i)} = 1 $$
\end{itemize}
Thus, $\langle\psi_{i}|\psi_{j}\rangle=\delta_{i,j}$.
}

\pr{ 3.4.1 (c)}{
Prove that any decomposition $I=\sum_{i}P_{i}$ of the identity operator on a Hilbert space of dimension $\mathbf{N}$ into a sum of nonzero projectors $P_{i}$ can have at most $\mathbf{N}$ terms in the sum.
}

\sol{
Let $\mathcal{H}$ be the Hilbert space with dimension $N$.
\begin{enumerate}
    \item The **Rank** of an orthogonal sum of operators is the sum of their ranks. Since $P_i$ are orthogonal projectors ($P_i P_j=0$ for $i \ne j$), we have:
    $$ \mathrm{rank}(I) = \mathrm{rank}\left(\sum_{i} P_{i}\right) = \sum_{i} \mathrm{rank}(P_{i}) $$
    \item The rank of the identity operator $I$ on an $N$-dimensional space is $N$: $\mathrm{rank}(I) = N$.
    \item Since each $P_i$ is a \textbf{nonzero} projector, its rank is at least 1: $\mathrm{rank}(P_i) \ge 1$.
    \item Let $M$ be the number of terms in the sum. Substituting the bounds:
    $$ N = \sum_{i=1}^{M} \mathrm{rank}(P_{i}) \ge \sum_{i=1}^{M} 1 = M $$
\end{enumerate}
Therefore, the number of nonzero orthogonal projectors $M$ cannot exceed the dimension of the space $N$. The decomposition can have at most $\mathbf{N}$ terms in the sum ($M \le N$).
}

\subsection*{Exercise 3.4.2: Equivalence of Observables}

\pr{ 3.4.2}{
Show that measuring the observable $|1\rangle\langle 1|$ is equivalent to measuring the observable $\mathbf{Z}$ up to a relabelling of the measurement outcomes.
}

\sol{
Two observables are equivalent (up to relabelling) if they have the exact same set of **projectors** onto their eigenspaces, as these projectors determine the physical action (state collapse) and probabilities of measurement.

\textbf{1. Observable $M_1 = |1\rangle\langle 1|$:}
$$ M_1 = \begin{pmatrix} 0 & 0 \\ 0 & 1 \end{pmatrix} $$
\begin{itemize}
    \item \textbf{Eigenvalues (Outcomes):} $\lambda_0 = 0$ and $\lambda_1 = 1$.
    \item \textbf{Projectors:}
        \begin{itemize}
            \item For outcome 0 (eigenvector $|0\rangle$): $P_0 = |0\rangle\langle 0|$.
            \item For outcome 1 (eigenvector $|1\rangle$): $P_1 = |1\rangle\langle 1|$.
        \end{itemize}
\end{itemize}

\textbf{2. Observable $M_2 = Z$ (Pauli $Z$):}
$$ Z = \begin{pmatrix} 1 & 0 \\ 0 & -1 \end{pmatrix} $$
\begin{itemize}
    \item \textbf{Eigenvalues (Outcomes):} $\lambda'_1 = 1$ and $\lambda'_{-1} = -1$.
    \item \textbf{Projectors:}
        \begin{itemize}
            \item For outcome 1 (eigenvector $|0\rangle$): $P'_1 = |0\rangle\langle 0|$.
            \item For outcome $-1$ (eigenvector $|1\rangle$): $P'_{-1} = |1\rangle\langle 1|$.
        \end{itemize}
\end{itemize}
The set of projectors for $M_1$ is $\mathcal{P}_1 = \{ |0\rangle\langle 0|, |1\rangle\langle 1| \}$.
The set of projectors for $M_2$ is $\mathcal{P}_2 = \{ |0\rangle\langle 0|, |1\rangle\langle 1| \}$.
Since $\mathcal{P}_1 = \mathcal{P}_2$, the physical collapse of the state is identical for both measurements. The only difference is the classical label recorded:
$$ \text{Outcome } 0 \text{ from } M_1 \iff \text{Outcome } 1 \text{ from } Z $$
$$ \text{Outcome } 1 \text{ from } M_1 \iff \text{Outcome } -1 \text{ from } Z $$
Thus, the measurements are equivalent up to a relabelling of the outcomes.
}

\section{Quantum Gates and Circuits}
\setcounter{section}{4} % Set section counter to 4 for Exercise 4.x.x
\subsection*{Exercise 4.2.3: Rotation Gate Conjugation and Euler Decomposition}

\pr{ 4.2.3 (a)}{
Prove $XR_{y}(\theta)X=R_{y}(-\theta)$ and $XR_{z}(\theta)X=R_{z}(-\theta)$.
}

\sol{
The matrices are: $X = \begin{pmatrix} 0 & 1 \\ 1 & 0 \end{pmatrix}$, $R_y(\theta) = \begin{pmatrix} \cos(\theta/2) & -\sin(\theta/2) \\ \sin(\theta/2) & \cos(\theta/2) \end{pmatrix}$, and $R_z(\theta) = \begin{pmatrix} e^{-i\theta/2} & 0 \\ 0 & e^{i\theta/2} \end{pmatrix}$.

\textbf{Proof 1: $XR_{y}(\theta)X=R_{y}(-\theta)$}
\begin{align*}
XR_{y}(\theta)X &= \begin{pmatrix} 0 & 1 \\ 1 & 0 \end{pmatrix} \begin{pmatrix} \cos(\theta/2) & -\sin(\theta/2) \\ \sin(\theta/2) & \cos(\theta/2) \end{pmatrix} \begin{pmatrix} 0 & 1 \\ 1 & 0 \end{pmatrix} \\
&= \begin{pmatrix} \sin(\theta/2) & \cos(\theta/2) \\ \cos(\theta/2) & -\sin(\theta/2) \end{pmatrix} \begin{pmatrix} 0 & 1 \\ 1 & 0 \end{pmatrix} \\
&= \begin{pmatrix} \cos(\theta/2) & \sin(\theta/2) \\ -\sin(\theta/2) & \cos(\theta/2) \end{pmatrix}
\end{align*}
The definition of $R_{y}(-\theta)$ is:
$$ R_{y}(-\theta) = \begin{pmatrix} \cos(-\theta/2) & -\sin(-\theta/2) \\ \sin(-\theta/2) & \cos(-\theta/2) \end{pmatrix} = \begin{pmatrix} \cos(\theta/2) & \sin(\theta/2) \\ -\sin(\theta/2) & \cos(\theta/2) \end{pmatrix} $$
Since the results are equal, the identity is proven.

\textbf{Proof 2: $XR_{z}(\theta)X=R_{z}(-\theta)$}
\begin{align*}
XR_{z}(\theta)X &= \begin{pmatrix} 0 & 1 \\ 1 & 0 \end{pmatrix} \begin{pmatrix} e^{-i\theta/2} & 0 \\ 0 & e^{i\theta/2} \end{pmatrix} \begin{pmatrix} 0 & 1 \\ 1 & 0 \end{pmatrix} \\
&= \begin{pmatrix} 0 & e^{i\theta/2} \\ e^{-i\theta/2} & 0 \end{pmatrix} \begin{pmatrix} 0 & 1 \\ 1 & 0 \end{pmatrix} \\
&= \begin{pmatrix} e^{i\theta/2} & 0 \\ 0 & e^{-i\theta/2} \end{pmatrix}
\end{align*}
The definition of $R_{z}(-\theta)$ is:
$$ R_{z}(-\theta) = \begin{pmatrix} e^{-i(-\theta)/2} & 0 \\ 0 & e^{i(-\theta)/2} \end{pmatrix} = \begin{pmatrix} e^{i\theta/2} & 0 \\ 0 & e^{-i\theta/2} \end{pmatrix} $$
Since the results are equal, the identity is proven.
}

\pr{ 4.2.3 (b)}{
Prove **Corollary 4.2.1**: Any single-qubit unitary operator $U$ can be written as $U=e^{i\alpha} A X B X C$, where $A, B, C$ are single-qubit unitaries (from hint: $A\equiv R_{z}(\beta)R_{y}(\gamma/2)$, $B\equiv R_{y}(-\gamma/2)R_{z}(-(\delta+\beta)/2)$ and $C\equiv R_{z}((\delta-\beta)/2)$).
}

\sol{
Any unitary operator $U \in \mathrm{U}(2)$ can be decomposed into a global phase $e^{i\alpha}$ and a special unitary operator $U_{\mathrm{SU}(2)}$ using the Euler decomposition:
$$ U_{\mathrm{SU}(2)} = R_{z}(\beta)R_{y}(\gamma)R_{z}(\delta) $$
We are asked to prove $U_{\mathrm{SU}(2)} = A X B X C$ using the provided definitions for $A$, $B$, and $C$. We substitute $A, B, C$ into the expression $A X B X C$:
$$ A X B X C = \left(R_{z}(\beta)R_{y}(\gamma/2)\right) X \left(R_{y}(-\gamma/2)R_{z}(-(\delta+\beta)/2)\right) X \left(R_{z}((\delta-\beta)/2)\right) $$
First, use the identity $R_z(\theta)X = X R_z(-\theta)$, derived from part (a):
$$ R_{z}(-(\delta+\beta)/2) X = X R_{z}((\delta+\beta)/2) $$
Substitute this back:
\begin{align*}
A X B X C &= R_{z}(\beta)R_{y}(\gamma/2) X R_{y}(-\gamma/2) \left(\mathbf{X R_{z}((\delta+\beta)/2)}\right) R_{z}((\delta-\beta)/2) \\
&= R_{z}(\beta)R_{y}(\gamma/2) \mathbf{X R_{y}(-\gamma/2) X} R_{z}\left(\frac{\delta+\beta}{2} + \frac{\delta-\beta}{2}\right)
\end{align*}
Next, use the identity $X R_{y}(-\theta) X = R_{y}(\theta)$ (from part a with $\theta \to -\theta$):
$$ X R_{y}(-\gamma/2) X = R_{y}(\gamma/2) $$
Substitute this identity and simplify the $R_z$ product:
\begin{align*}
A X B X C &= R_{z}(\beta)R_{y}(\gamma/2) \mathbf{R_{y}(\gamma/2)} R_{z}\left(\frac{2\delta}{2}\right) \\
&= R_{z}(\beta)R_{y}(\gamma)R_{z}(\delta) \\
&= U_{\mathrm{SU}(2)}
\end{align*}
Therefore, $\mathbf{U=e^{i\alpha} A X B X C}$ is proven.
}

\newpage
\subsection*{Exercise 4.2.4: CNOT in Different Bases}

\pr{ 4.2.4 (a)}{
Describe the effect of the **CNOT gate** with respect to the basis $B_{1}=\{|0\rangle|+\rangle , |0\rangle|-\rangle , |1\rangle|+\rangle , |1\rangle|-\rangle\}$. Express your answers in Dirac notation and matrix notation.
}

\sol{
The CNOT gate acts on $|c\rangle|t\rangle$ as $CNOT|c\rangle|t\rangle = |c\rangle|t \oplus c\rangle$. The basis $B_1$ uses the computational basis for the control qubit ($|0\rangle, |1\rangle$) and the Hadamard basis for the target qubit ($|\pm\rangle = \frac{|0\rangle \pm |1\rangle}{\sqrt{2}}$).

\textbf{Dirac Notation:}
\begin{itemize}
    \item \textbf{Control $|0\rangle$:} The target is unchanged (Target gate is $I$).
    $$ CNOT|0\rangle|\pm\rangle = |0\rangle I|\pm\rangle = |0\rangle|\pm\rangle $$
    \item \textbf{Control $|1\rangle$:} The target is flipped (Target gate is $X$).
    $$ CNOT|1\rangle|\pm\rangle = |1\rangle X|\pm\rangle $$
    Since $X|+\rangle = |+\rangle$ and $X|-\rangle = -|-\rangle$ (eigenstates of $X$ are $|\pm\rangle$ with eigenvalues $\pm 1$):
    $$ CNOT|1\rangle|+\rangle = |1\rangle|+\rangle $$
    $$ CNOT|1\rangle|-\rangle = -|1\rangle|-\rangle $$
\end{itemize}

\textbf{Matrix Notation ($U_{B_1}$):}
The basis order is $B_1 = \{|0+\rangle, |0-\rangle, |1+\rangle, |1-\rangle\}$.
The mapping is:
$$ |0+\rangle \to 1 \cdot |0+\rangle $$
$$ |0-\rangle \to 1 \cdot |0-\rangle $$
$$ |1+\rangle \to 1 \cdot |1+\rangle $$
$$ |1-\rangle \to -1 \cdot |1-\rangle $$
Since the basis vectors are eigenvectors of $CNOT$ in this basis, the matrix is diagonal with the eigenvalues $\{1, 1, 1, -1\}$:
$$ U_{B_1} = \begin{pmatrix} 1 & 0 & 0 & 0 \\ 0 & 1 & 0 & 0 \\ 0 & 0 & 1 & 0 \\ 0 & 0 & 0 & -1 \end{pmatrix} $$
}

\pr{ 4.2.4 (b)}{
Describe the effect of the **CNOT gate** with respect to the basis $B_{2}=\{|+\rangle|+\rangle, |+\rangle|-\rangle, |-\rangle|+\rangle, |-\rangle|-\rangle\}$. Express your answers in Dirac notation and matrix notation.
}

\sol{
The basis $B_2$ is the product of Hadamard bases for both qubits.
The CNOT gate in the Hadamard basis is equivalent to the CNOT gate with the control and target qubits swapped in the computational basis ($CNOT_{21}$), as:
$$ (H \otimes H) CNOT_{12} (H \otimes H) = CNOT_{21} $$
where $CNOT_{21}$ flips the first qubit (target) if the second qubit (control) is $|1\rangle$.

\textbf{Dirac Notation (using $CNOT_{21}$ equivalence):}
\begin{itemize}
    \item \textbf{Control $|+\rangle$ (qubit 1):} $|\pm\rangle$ is the control. CNOT is flipped, so qubit 2 is the control.
    We use the definition of $CNOT_{12}$:
    $$ CNOT|+\rangle|+\rangle = CNOT \frac{1}{2}(|00\rangle + |01\rangle + |10\rangle + |11\rangle) $$
    $$ = \frac{1}{2}(|00\rangle + |01\rangle + |11\rangle + |10\rangle) = |+\rangle|+\rangle $$
    $$ CNOT|+\rangle|-\rangle = CNOT \frac{1}{2}(|00\rangle - |01\rangle + |10\rangle - |11\rangle) $$
    $$ = \frac{1}{2}(|00\rangle - |01\rangle + |11\rangle - |10\rangle) = |-\rangle|-\rangle $$
    $$ CNOT|-\rangle|+\rangle = CNOT \frac{1}{2}(|00\rangle + |01\rangle - |10\rangle - |11\rangle) $$
    $$ = \frac{1}{2}(|00\rangle + |01\rangle - |11\rangle - |10\rangle) = |-\rangle|+\rangle $$
    $$ CNOT|-\rangle|-\rangle = CNOT \frac{1}{2}(|00\rangle - |01\rangle - |10\rangle + |11\rangle) $$
    $$ = \frac{1}{2}(|00\rangle - |01\rangle - |11\rangle + |10\rangle) = |+\rangle|-\rangle $$
\end{itemize}
\textbf{The mappings are:}
$$ \begin{aligned} CNOT|+\rangle|+\rangle &= |+\rangle|+\rangle \\ CNOT|+\rangle|-\rangle &= |-\rangle|-\rangle \\ CNOT|-\rangle|+\rangle &= |-\rangle|+\rangle \\ CNOT|-\rangle|-\rangle &= |+\rangle|-\rangle \end{aligned} $$

\textbf{Matrix Notation ($U_{B_2}$):}
The basis order is $B_2 = \{|++\rangle, |+-\rangle, |-+\rangle, |--\rangle\}$.
\begin{itemize}
    \item $|++\rangle \to |++\rangle$
    \item $|+-\rangle \to |--\rangle$ (swaps with the 4th element's position)
    \item $|-+\rangle \to |-+\rangle$
    \item $|--\rangle \to |+-\rangle$ (swaps with the 2nd element's position)
\end{itemize}
The resulting matrix is:
$$ U_{B_2} = \begin{pmatrix} 1 & 0 & 0 & 0 \\ 0 & 0 & 0 & 1 \\ 0 & 0 & 1 & 0 \\ 0 & 1 & 0 & 0 \end{pmatrix} $$
}
\newpage
\section{Measurement in Quantum Computing}
\setcounter{section}{5} % Set section counter to 5 for Exercise 5.x.x
\subsection*{Exercise 5.2.1: Bell State Decomposition Identity}

\pr{ 5.2.1}{
Prove that
$$|\psi\rangle|\beta_{00}\rangle=\frac{1}{2}|\beta_{00}\rangle|\psi\rangle+\frac{1}{2}|\beta_{01}\rangle(X|\psi\rangle)+\frac{1}{2}|\beta_{10}\rangle(Z|\psi\rangle)+\frac{1}{2}|\beta_{11}\rangle(XZ|\psi\rangle).$$
}
\sol{
We label the three registers as 1,2,3: register 1 holds the arbitrary single-qubit state $\ket{\psi}_1$, and registers 2--3 hold the Bell state $\ket{\beta_{00}}_{23}$. The Bell-basis resolution of identity on registers 1 and 2 is given by:
\[
I_{12}=\sum_{a,b\in\{0,1\}}\ket{\beta_{ab}}_{12}\bra{\beta_{ab}}.
\]
Thus
\[
\ket{\psi}_1\ket{\beta_{00}}_{23}
= \sum_{a,b}\ket{\beta_{ab}}_{12}\,\Big(\bra{\beta_{ab}}_{12}\,(\ket{\psi}_1\ket{\beta_{00}}_{23})\Big).
\]
We evaluate the overlap
\[
\mathcal{C}_{ab}:=\bra{\beta_{ab}}_{12}\,(\ket{\psi}_1\ket{\beta_{00}}_{23}).
\]
Use $\ket{\beta_{ab}}_{12}=(I\otimes X^{b}Z^{a})\ket{\beta_{00}}_{12}$, so
\[
\bra{\beta_{ab}}_{12}=\bra{\beta_{00}}_{12}\,(I\otimes Z^{a}X^{b}).
\]
Therefore
\[
\mathcal{C}_{ab}
= \bra{\beta_{00}}_{12}\,(I\otimes Z^{a}X^{b})\big(\ket{\psi}_1\ket{\beta_{00}}_{23}\big).
\]
Now use the maximally-entangled identity $(I\otimes M)\ket{\beta_{00}}=(M^{\mathsf T}\otimes I)\ket{\beta_{00}}$ (and $X^{\mathsf T}=X,\;Z^{\mathsf T}=Z$) to move $Z^{a}X^{b}$ onto register 1:
\[
(I_1\otimes Z_2^{a}X_2^{b})\big(\ket{\psi}_1\ket{\beta_{00}}_{23}\big)
= \big((X^{b}Z^{a}\ket{\psi})_1\big)\otimes\ket{\beta_{00}}_{23}.
\]
Hence
\[
\mathcal{C}_{ab}
= \bra{\beta_{00}}_{12}\,\big((X^{b}Z^{a}\ket{\psi})_1\otimes\ket{\beta_{00}}_{23}\big).
\]
Finally use the overlap identity $\bra{\beta_{00}}_{12}\big(\ket{\phi}_1\otimes\ket{\beta_{00}}_{23}\big)=\tfrac{1}{2}\ket{\phi}_3$ (valid for any single-qubit $\ket\phi$). With $\ket\phi=X^{b}Z^{a}\ket\psi$ we get
\[
\mathcal{C}_{ab}=\tfrac{1}{2}\,(X^{b}Z^{a}\ket\psi)_3.
\]
Substituting back into the sum gives
\[
\ket{\psi}_1\ket{\beta_{00}}_{23}
=\tfrac{1}{2}\sum_{a,b\in\{0,1\}}\ket{\beta_{ab}}_{12}\,(X^{b}Z^{a}\ket\psi)_3,
\]
which, when expanded term-by-term, is exactly
\[
\ket{\psi}_1\ket{\beta_{00}}_{23}
=\tfrac{1}{2}\ket{\beta_{00}}_{12}\ket{\psi}_3
+\tfrac{1}{2}\ket{\beta_{01}}_{12}(X\ket{\psi})_3
+\tfrac{1}{2}\ket{\beta_{10}}_{12}(Z\ket{\psi})_3
+\tfrac{1}{2}\ket{\beta_{11}}_{12}(XZ\ket{\psi})_3.
\]
}

\section{Quantum Algorithms}
\setcounter{section}{6} % Set section counter to 6 for Exercise 6.x.x
\subsection*{Exercise 6.4.1: Multiqubit Hadamard Transform}

\pr{ 6.4.1}{
Prove that
\begin{multline*}
\left(\frac{|0\rangle+(-1)^{x_{1}}|1\rangle}{\sqrt{2}}\right)\left(\frac{|0\rangle+(-1)^{x_{2}}|1\rangle}{\sqrt{2}}\right)\cdots \\
\cdots \left(\frac{|0\rangle+(-1)^{x_{n}}|1\rangle}{\sqrt{2}}\right)
= \frac{1}{\sqrt{2^{n}}}\sum_{z_{1}z_{2}\ldots z_{n}\in\{0,1\}^{n}}(-1)^{x_{1}z_{1}+x_{2}z_{2}+\cdots+x_{n}z_{n}} \\
{}\times |z_{1}\rangle|z_{2}\rangle\cdots|z_{n}\rangle.
\end{multline*}
}

\sol{
We prove the identity by expanding each factor and then taking the tensor product.

For each \(k\):
\[
\frac{\ket{0}+(-1)^{x_k}\ket{1}}{\sqrt{2}}
= \frac{1}{\sqrt{2}}\sum_{z_k\in\{0,1\}}(-1)^{x_k z_k}\ket{z_k},
\]
because when $z_k=0$ the term equals $\tfrac1{\sqrt2}\ket0$ and when $z_k=1$ it equals $\tfrac{(-1)^{x_k}}{\sqrt2}\ket1$.

Now take the tensor product over $k=1,\dots,n$:
\[
\bigotimes_{k=1}^n\frac{\ket{0}+(-1)^{x_k}\ket{1}}{\sqrt{2}}
= \left(\prod_{k=1}^n\frac{1}{\sqrt{2}}\right)
\bigotimes_{k=1}^n\sum_{z_k\in\{0,1\}}(-1)^{x_k z_k}\ket{z_k}
= \frac{1}{\sqrt{2^n}}\bigotimes_{k=1}^n\sum_{z_k\in\{0,1\}}(-1)^{x_k z_k}\ket{z_k}.
\]

Expanding the tensor product of sums gives a single sum over all $n$-bit strings \(z=(z_1,\dots,z_n)\):
\[
\bigotimes_{k=1}^n\sum_{z_k\in\{0,1\}}(-1)^{x_k z_k}\ket{z_k}
= \sum_{z_1,\ldots,z_n\in\{0,1\}} \left(\prod_{k=1}^n(-1)^{x_k z_k}\right)\ket{z_1}\ket{z_2}\cdots\ket{z_n}.
\]
Since
\[
\prod_{k=1}^n(-1)^{x_k z_k} = (-1)^{\sum_{k=1}^n x_k z_k},
\]
we obtain
\[
\bigotimes_{k=1}^n\frac{\ket{0}+(-1)^{x_k}\ket{1}}{\sqrt{2}}
= \frac{1}{\sqrt{2^n}}\sum_{z\in\{0,1\}^n}(-1)^{\sum_{k=1}^n x_k z_k}\ket{z_1}\ket{z_2}\cdots\ket{z_n},
\]
which is the desired identity.
}


\subsection*{Exercise 6.5.1: Action of Multiqubit Hadamard on a Superposition}

\pr{ 6.5.1}{
Let $x,y\in\{0,1\}^{n}$ and let $s=x\oplus y$. Show that
$$H^{\otimes n}(\frac{1}{\sqrt{2}}|x\rangle+\frac{1}{\sqrt{2}}|y\rangle)=\frac{1}{\sqrt{2^{n-1}}}\sum_{z\in\{s\}^{\perp}}(-1)^{x\cdot z}|z\rangle.$$
}

\sol{
\textbf{1. Apply $H^{\otimes n}$ to the Superposition}
We use the multi-qubit Hadamard formula $H^{\otimes n}|w\rangle = \frac{1}{\sqrt{2^n}}\sum_{z}(-1)^{w\cdot z}|z\rangle$:
\begin{align*}
H^{\otimes n}\left(\frac{1}{\sqrt{2}}|x\rangle+\frac{1}{\sqrt{2}}|y\rangle\right) &= \frac{1}{\sqrt{2}} H^{\otimes n}|x\rangle + \frac{1}{\sqrt{2}} H^{\otimes n}|y\rangle \\
&= \frac{1}{\sqrt{2}} \left( \frac{1}{\sqrt{2^n}}\sum_{z}(-1)^{x\cdot z}|z\rangle \right) + \frac{1}{\sqrt{2}} \left( \frac{1}{\sqrt{2^n}}\sum_{z}(-1)^{y\cdot z}|z\rangle \right) \\
&= \frac{1}{\sqrt{2^{n+1}}} \sum_{z \in \{0,1\}^n} \left( (-1)^{x\cdot z} + (-1)^{y\cdot z} \right) |z\rangle
\end{align*}

\textbf{2. Analyze the Coefficient $C_z = (-1)^{x\cdot z} + (-1)^{y\cdot z}$}
Since $s = x \oplus y$, the dot product $s \cdot z$ relates $x \cdot z$ and $y \cdot z$ modulo 2: $s \cdot z = (x \cdot z) \oplus (y \cdot z) \pmod{2}$.
\begin{itemize}
    \item \textbf{Case 1: $z \in \{s\}^{\perp}$ (where $s \cdot z = 0 \pmod{2}$)}
    If $s \cdot z = 0$, then $x \cdot z$ and $y \cdot z$ have the same parity, so $(-1)^{x\cdot z} = (-1)^{y\cdot z}$.
    $$ C_z = (-1)^{x\cdot z} + (-1)^{x\cdot z} = 2 (-1)^{x\cdot z} $$
    \item \textbf{Case 2: $z \notin \{s\}^{\perp}$ (where $s \cdot z = 1 \pmod{2}$)}
    If $s \cdot z = 1$, then $x \cdot z$ and $y \cdot z$ have opposite parity, so $(-1)^{y\cdot z} = - (-1)^{x\cdot z}$.
    $$ C_z = (-1)^{x\cdot z} + (-(-1)^{x\cdot z}) = 0 $$
\end{itemize}

\textbf{3. Final Substitution}
Only the terms where $z \in \{s\}^{\perp}$ survive, which is a subspace of dimension $n-1$:
\begin{align*}
H^{\otimes n}\left(\frac{1}{\sqrt{2}}|x\rangle+\frac{1}{\sqrt{2}}|y\rangle\right) &= \frac{1}{\sqrt{2^{n+1}}} \sum_{z \in \{s\}^{\perp}} \left( 2 (-1)^{x\cdot z} \right) |z\rangle \\
&= \frac{2}{\sqrt{2^{n} \cdot 2}} \sum_{z \in \{s\}^{\perp}} (-1)^{x\cdot z} |z\rangle \\
&= \frac{\sqrt{4}}{\sqrt{2^{n} \cdot 2}} \sum_{z \in \{s\}^{\perp}} (-1)^{x\cdot z} |z\rangle \\
&= \frac{\sqrt{2}}{\sqrt{2^{n}}} \sum_{z \in \{s\}^{\perp}} (-1)^{x\cdot z} |z\rangle \\
&= \frac{1}{\sqrt{2^{n-1}}} \sum_{z \in \{s\}^{\perp}} (-1)^{x\cdot z} |z\rangle
\end{align*}
}

\end{document}