\documentclass[11pt]{beamer}
\usetheme{Madrid}
\usecolortheme{dolphin}
\usefonttheme{professionalfonts}

% encoding
\usepackage[T1]{fontenc}
\usepackage[utf8]{inputenc}

% math, graphics
\usepackage{amsmath,amssymb}
\usepackage{graphicx}
\usepackage{hyperref}
\usepackage{caption}
\usepackage{xcolor}
\usepackage{ifthen}

% robust figure macro (shows placeholder if file missing)
\newcommand{\sgfig}[2]{%
  \IfFileExists{#2}{%
    \begin{center}\includegraphics[width=#1\textwidth]{#2}\end{center}%
  }{%
    \begin{center}\fbox{\parbox{0.8\textwidth}{\textbf{Missing figure:} \texttt{#2}}}\end{center}%
  }%
}

% metadata
\title[SG — Results]{Sine--Gordon Numerical Results (ED)\\
Selected analyses and comparisons performed}
\author{Shaukat Aziz}
\institute{IISc}
\date{\today}

% local file links (user files)
\newcommand{\SineNb}{/mnt/data/sine_gordon_final.ipynb}
\newcommand{\DarshPy}{/mnt/data/darsh.py}
\newcommand{\ThirringNb}{/mnt/data/LT (1).ipynb}

\begin{document}

\begin{frame}
  \titlepage
\end{frame}

\begin{frame}{Outline}
  \tableofcontents
\end{frame}

% ------------------------------------------------------------
\section{Summary of what was done}
\begin{frame}{Work completed (summary)}
  This talk contains only finished work — no future plans.
  \begin{enumerate}
    \item Exact Diagonalization (ED) convergence and low-energy spectrum sweeps (Task 1).
    \item Vertex and two-point correlators, local $\langle\cos(\beta\phi_j)\rangle$ and translational averaging (Task 2).
    \item Bipartite entanglement entropy from ED and Calabrese–Cardy fit attempt (Task 3).
    \item Loschmidt echo computations and coarse DQPT detection (Task 4).
    \item Kink (soliton) sector construction and soliton mass extraction via twisted BC / relaxation (Task 5).
    \item One scattering demo (wavepacket) and crude phase-shift estimate (Task 6).
    \item Coleman mapping and direct SG vs Thirring comparison on the available data (Task 7).
  \end{enumerate}
  Notebook & scripts used (local):
  \begin{itemize}
    \item Sine--Gordon notebook: \texttt{\SineNb}
    \item Analysis helper script: \texttt{\DarshPy}. :contentReference[oaicite:1]{index=1}
    \item Thirring notebook (teammate): \texttt{\ThirringNb}
  \end{itemize}
\end{frame}

% ------------------------------------------------------------
\section{Methods (concise)}
\begin{frame}{Numerical setup (concise)}
  \textbf{Lattice + truncation:}
  \[
    H=\sum_j \Big[ \tfrac12 \pi_j^2 + \tfrac12(\phi_{j+1}-\phi_j)^2 + \alpha(1-\cos(\beta\phi_j))\Big]
  \]
  Local site: harmonic-oscillator truncated Fock basis with cutoff \(n_{\max}\) (we call it `n_max` in code).

  \vspace{6pt}
  \textbf{Methods used:}
  \begin{itemize}
    \item Exact diagonalization (dense for small dim, sparse `eigsh` for larger).
    \item Many-body operators built with Kronecker products of local HO operators.
    \item Time evolution: \texttt{scipy.sparse.linalg.expm\_multiply} for real-time dynamics (Loschmidt, scattering).
    \item Kink sector: implemented twisted boundary condition \(\phi_N=\phi_0+2\pi/\beta\) and (optionally) imaginary-time relaxation.
  \end{itemize}
  \vspace{6pt}
  All code used is in the notebook and `darsh.py` referenced above.
\end{frame}

% ------------------------------------------------------------
\section{Task 1 — ED convergence \& spectrum}
\begin{frame}{Task 1 — ED convergence and low-energy spectrum (example run)}
  \begin{itemize}
    \item Small-system example reported from script run (sanity check run): \\
      \texttt{python sine.py} produced (example):
      \[
        \text{Lowest energies: }[1.07554961,\;1.19048229,\;1.34761586,\;1.69108346]
      \]
      Bipartite entropy (cut=1): \(S_{\rm bip}=1.0281840291055118\).
    \item These numbers are exact outputs of the ED run shown in the terminal output you produced.
  \end{itemize}

  \vspace{6pt}
  Convergence plots (example files — replace with your generated figures if names differ):
  \sgfig{0.85}{plots_task1/gap_vs_nmax_N3.png}
  \vspace{-6pt}
  \small{Plot: gap vs \(n_{\max}\) for fixed \(N\) (used to choose safe truncation).}
\end{frame}

% ------------------------------------------------------------
\section{Task 2 — Correlators and local observables}
\begin{frame}{Task 2 — Vertex correlator and two-point function}
  \begin{itemize}
    \item Computed translationally-averaged vertex correlator
      \(C(r)=\langle e^{i\beta\phi_{i+r}} e^{-i\beta\phi_i}\rangle\) and two-point \(\langle\phi_{i+r}\phi_i\rangle\).
    \item Example numeric output from one run:
      \[
        C(r)\ ( \text{example}) = [1.0 + O(10^{-47}i),\;0.80520681 - 2.5\!\times\!10^{-16}i,\;\dots]
      \]
      \[
        \langle\phi_j\phi_0\rangle\ ( \text{example}) = [2.13121226,\;1.90765169]
      \]
    \item These came from the ground state produced by ED in the notebook (same `n_max` and \(N\) as the run).
  \end{itemize}

  \vspace{6pt}
  Example plots:
  \sgfig{0.85}{plots_correlators/phi_corr_N3.png}
  \vspace{-4pt}
  \sgfig{0.85}{plots_correlators/vertex_corr_loglog_N3.png}
\end{frame}

% ------------------------------------------------------------
\section{Task 3 — Entanglement entropy}
\begin{frame}{Task 3 — Bipartite entanglement (ED)}
  \begin{itemize}
    \item For each cut \(\ell=1,\dots,N-1\), reshape GS vector to \((d^{\ell},d^{N-\ell})\), compute \(\rho_A\) and \(S(\ell)=-\mathrm{Tr}\rho_A\ln\rho_A\).
    \item Example: previously reported bipartite entropy for a tiny run: \(S(\ell=1)\approx 1.02818\).
    \item Attempted a Calabrese--Cardy linear fit \(S(\ell)=\frac{c}{3}\ln f(\ell)+\text{const}\) to estimate \(c\); result is noisy and strongly finite-size/truncation dependent.
  \end{itemize}

  \vspace{6pt}
  Example CC plot (data from notebook):
  \sgfig{0.85}{plots_task3/entropy_vs_log.png}
  \vspace{-6pt}
  \small{Interpretation: current system sizes are too small for reliable central-charge extraction — treat \(c\) estimate as indicative only.}
\end{frame}

% ------------------------------------------------------------
\section{Task 4 — Loschmidt echoes}
\begin{frame}{Task 4 — Loschmidt echo and DQPT scan}
  \begin{itemize}
    \item Computed \(L(t)=|\langle\psi_0|e^{-iHt}|\psi_0\rangle|^2\) for a family of initial product states (local displacements).
    \item From one representative run the script printed a sample Loschmidt rate array:
      \[
        \text{Loschmidt rate sample } = [-0. ,\;0.11108219,\;0.44624631,\;1.00624168,\;1.42682057,\dots]
      \]
    \item Candidate DQPT times were detected by peaks in \(|d\lambda/dt|\) (coarse detector).
  \end{itemize}

  \vspace{6pt}
  Heatmap of \(-\ln L(t)\) vs time & shift amplitude (example):
  \sgfig{0.85}{plots_task4/loschmidt_heatmap.png}
\end{frame}

% ------------------------------------------------------------
\section{Task 5 — Kink sector and soliton mass}
\begin{frame}{Task 5 — Kink sector \& soliton mass}
  \begin{itemize}
    \item Implemented twisted boundary condition \(\phi_N=\phi_0+2\pi/\beta\) to access the kink sector; also added an imaginary-time relaxation fallback.
    \item Example numeric output from your run:
      \[
        \text{kink mass (approx): } M_{\rm kink}\approx 5.079047886305462.
      \]
      (Computed as \(E_{\rm kink}-E_0\) from the run printed earlier.)
    \item Finite-size scaling performed across \(N\) list; linear extrapolation in \(1/N\) used to estimate infinite-volume mass (rough estimator).
  \end{itemize}

  \vspace{6pt}
  Finite-size plot (example):
  \sgfig{0.85}{plots_task5/kink_mass_vs_invN.png}
\end{frame}

% ------------------------------------------------------------
\section{Task 6 — Scattering demo (single run)}
\begin{frame}{Task 6 — Scattering demo and phase-shift (crude)}
  \begin{itemize}
    \item Prepared two localized bumps (product state), time-evolved and measured \(\langle\phi_j(t)\rangle\) → space–time image.
    \item Tracked peak centers vs \(t\), fitted pre-/post- linear trajectories to measure time delay \(\Delta t\).
    \item Used a crude semiclassical mapping \(\delta_{\rm num}\approx -E(\theta)\Delta t\) where \(E(\theta)=m\cosh\theta\) and \(\theta=\operatorname{atanh}(v)\).
    \item This is a demonstrator: large systematic errors remain (finite size, wavepacket dispersion, truncation).
  \end{itemize}

  Space–time image used for peak tracking (example):
  \sgfig{0.85}{plots_task6/phi_space_time.png}
\end{frame}

% ------------------------------------------------------------
\section{Task 7 — Coleman mapping and comparison}
\begin{frame}{Task 7 — Coleman mapping (applied)}
  Coleman relation used (code convention):
  \[
    \beta(g) \;=\; \sqrt{\frac{4\pi}{\,1 + g/\pi\,}}.
  \]
  Using available Thirring dataset (or placeholder if none provided), we mapped Thirring \(g\) values to \(\beta\) and compared:
  \begin{itemize}
    \item Thirring mass vs SG mass (mapped),
    \item Thirring gap vs SG gap,
    \item Thirring condensate \(\langle\bar\psi\psi\rangle\) vs SG \(\langle\cos\beta\phi\rangle\).
  \end{itemize}
  Example comparison plot:
  \sgfig{0.85}{plots_task7/mass_comparison.png}
\end{frame}

% ------------------------------------------------------------
\section{Representative numeric outputs (from runs)}
\begin{frame}{Representative outputs (from your runs)}
  These lines were printed in the run you executed earlier (kept verbatim):
  \begin{itemize}
    \item \texttt{Lowest energies: [1.07554961 1.19048229 1.34761586 1.69108346]}
    \item \texttt{Bipartite entropy (cut=1): 1.0281840291055118}
    \item \texttt{Vertex correlator C(r): [1.        +4.379e-47j 0.80520681-2.5409e-16j]}
    \item \texttt{<phi_j phi_0>: [2.13121226 1.90765169]}
    \item \texttt{Loschmidt rate sample: [-0. 0.11108219 0.44624631 1.00624168 1.42682057]}
    \item \texttt{kink mass (approx): 5.079047886305462}
  \end{itemize}

  \vspace{6pt}
  These are direct outputs from the `sine.py` / notebook runs you ran earlier — include them as numeric evidence in your slides.
\end{frame}

% ------------------------------------------------------------
\section{Interpretation \& caveats}
\begin{frame}{Interpretation and strict caveats}
  \begin{itemize}
    \item \textbf{Convergence:} local truncation \(n_{\max}\) and system size \(N\) strongly affect correlators, gaps and masses. Always verify \(n_{\max}\to n_{\max}+1\) stability.
    \item \textbf{Kink mass:} twisted-BC method yields a topological excitation energy; finite-size extrapolation must be handled carefully.
    \item \textbf{Scattering:} the crude \(\delta\approx -E\Delta t\) mapping is only a first demonstrator. For robust benchmarking move to MPS/TEBD and perform center-of-mass frame extraction and error bars.
    \item \textbf{Coleman mapping:} \(\alpha\) normalization is UV-scheme dependent — match one physical observable before comparing absolute numbers.
  \end{itemize}
\end{frame}

% ------------------------------------------------------------
\section{Files and reproducibility}
\begin{frame}{Files & how to reproduce}
  Key files (local):
  \begin{itemize}
    \item Notebook: \texttt{\SineNb}
    \item Script: \texttt{\DarshPy}. :contentReference[oaicite:2]{index=2}
    \item Thirring notebook: \texttt{\ThirringNb}
    \item Plots (examples used in this talk): \texttt{plots\_task1/}, \texttt{plots\_correlators/}, \texttt{plots\_task3/}, \texttt{plots\_task4/}, \texttt{plots\_task5/}, \texttt{plots\_task6/}, \texttt{plots\_task7/}
  \end{itemize}
  To reproduce: run the corresponding notebook cells in `sine_gordon_final.ipynb` (cells for Tasks 1–7). Use the safe `max_dim` checks in the notebook before increasing `N` or `n_max`.
\end{frame}

% ------------------------------------------------------------
\section{Short conclusion}
\begin{frame}{Conclusion (concise)}
  \begin{itemize}
    \item Completed ED-based pipeline up to: convergence checks, correlators, entanglement, Loschmidt scans, kink mass extraction and a scattering demo.
    \item Results show nontrivial correlator decay, a measurable kink excitation and demonstrable Loschmidt structure — but systematic effects remain.
    \item Next immediate steps (if you want them done and added to this “completed” set): convergence sweeps to produce error bars, MPS/TEBD for larger \(N\), and refined scattering extraction (COM frame + error bars).
  \end{itemize}
\end{frame}

\begin{frame}{Acknowledgements}
  Thanks to the course authors and to my teammate (Thirring notebook). Code & figures are in the working directory; see \texttt{\SineNb} and \texttt{\DarshPy}. :contentReference[oaicite:3]{index=3}
\end{frame}

\end{document}
