% Clean Beamer presentation (fixed titlepage and TOC)
\documentclass{beamer}
\usetheme{Madrid}
\usecolortheme{seahorse}
\usepackage{amsmath,amssymb}
\usepackage{graphicx}
\usepackage{hyperref}
\graphicspath{{./}}

\title[Terahertz quantum sensing]{Terahertz Quantum Sensing}
\subtitle{``Terahertz quantum sensing'' — M. Kutas et al., Sci. Adv. 2020}
\author[Your Name]{Prepared by: Your Name}
\date{August 26, 2025}

\begin{document}

% Title frame (use \frame{} with \titlepage)
\begin{frame}[plain]
  \titlepage
\end{frame}

% Table of contents (use only once)
\begin{frame}{Outline}
  \tableofcontents[hideallsubsections]
\end{frame}

\section{Motivation}
\begin{frame}{Why terahertz quantum sensing?}
  \begin{itemize}
    \item Terahertz (THz) carries material-specific contrast but detectors are inefficient.
    \item Use correlated photons: idler (THz) interacts with sample; signal (visible) is detected.
    \item Infer THz properties without direct THz detection.
  \end{itemize}
\end{frame}

\section{Key concepts}
\begin{frame}{Induced coherence without induced emission}
  \begin{itemize}
    \item SPDC produces signal (visible) + idler (THz) photon pairs.
    \item Two indistinguishable generation pathways interfere at the signal, dependent on idler path.
    \item Object transmissivity and idler phase modulate interference visibility.
  \end{itemize}
\end{frame}

\begin{frame}{Operator picture (intuition)}
  \begin{equation*}
    \hat a'_s = u\,\hat a_s + v\,\hat a_i^{\dagger},\qquad
    \hat a'_i = u\,\hat a_i + v\,\hat a_s^{\dagger},
  \end{equation*}
  with $|u|^2-|v|^2=1$. Low-gain: $|v|\ll1$.
\end{frame}

\section{Theory}
\begin{frame}{Signal rate (schematic)}
  \begin{equation}
    R(\theta_s) = N\int d\omega_i\,d\theta_i\;\big[1 + t(\theta_i)\cos(\phi_0 + \tfrac{\omega_i}{c}\Delta l_i)\big]
  \end{equation}
\end{frame}

\section{Experiment}
\begin{frame}{Setup (Fig.1)}
  \centering
  \includegraphics[width=0.85\linewidth]{fig-001.png}
  \\
  {\scriptsize Schematic (paper Fig.1).}
\end{frame}

\begin{frame}{Layout (Fig.2)}
  \centering
  \includegraphics[width=0.9\linewidth]{fig-000.png}
  \\
  {\scriptsize Experimental layout and key components.}
\end{frame}

\section{Results}
\begin{frame}{Interference and FFT (Fig.3)}
  \centering
  \includegraphics[width=0.9\linewidth]{fig-002.png}
  \\
  {\scriptsize FFT peak near 1.26 THz.}
\end{frame}

\begin{frame}{Thickness sensing (Fig.4)}
  \centering
  \includegraphics[width=0.9\linewidth]{fig-003.png}
  \\
  {\scriptsize Fitted envelope shift vs PTFE thickness.}
\end{frame}

\begin{frame}{Checks (Fig.5 & Fig.6)}
  \centering
  \includegraphics[width=0.48\linewidth]{fig-004.png}\hfill
  \includegraphics[width=0.48\linewidth]{fig-005.png}
  \\
  {\scriptsize Pump-power linearity and idler angular distribution.}
\end{frame}

\section{Conclusions}
\begin{frame}{Conclusions}
  \begin{itemize}
    \item Induced coherence enables THz sensing using visible detectors.
    \item Demonstrated thickness extraction and FFT signatures around 1.26 THz.
    \item Improvements: better idler collection, reduced loss, spectral control.
  \end{itemize}
\end{frame}

\begin{frame}{References}
  \footnotesize
  M. Kutas et al., "Terahertz quantum sensing," Sci. Adv. 6, eaaz8065 (2020).
\end{frame}

\end{document}
\documentclass{beamer}
\usetheme{Madrid}
\usecolortheme{seahorse}
\usepackage{amsmath,amssymb}
\usepackage{graphicx}
\usepackage{hyperref}
\graphicspath{{./}}

	itle{Terahertz Quantum Sensing}
\subtitle{``Terahertz quantum sensing'' — M. Kutas et al., Sci. Adv. 2020}
\author{Prepared by: Your Name}
\date{August 26, 2025}

\begin{document}

\begin{frame}
		itlepage
\end{frame}

\begin{frame}{Outline}
		ableofcontents
\end{frame}

\section{Motivation}
\begin{frame}{Why terahertz quantum sensing?}
	\begin{itemize}
		\item Terahertz (THz) region contains valuable material information but lacks sensitive detectors.
		\item Idea: generate correlated photon pairs; let the idler (THz) interact with the sample and detect the signal (visible).
		\item Benefit: access THz information while using efficient visible detectors.
	\end{itemize}
\end{frame}

\section{Key concepts}
\begin{frame}{Induced coherence without induced emission}
	\begin{itemize}
		\item SPDC in a nonlinear crystal creates signal (visible) + idler (THz) pairs.
		\item Two indistinguishable generation paths (first vs second pass) cause interference in the detected signal even if idler is undetected.
		\item Interference depends on idler path phase and object transmissivity — basis for sensing.
	\end{itemize}
\end{frame}

\begin{frame}{SPDC — operator picture (intuition)}
	\begin{equation*}
		\hat a'_s = u\,\hat a_s + v\,\hat a_i^{\dagger},\qquad
		\hat a'_i = u\,\hat a_i + v\,\hat a_s^{\dagger},
	\end{equation*}
	with $|u|^2-|v|^2=1$. Low-gain regime: $|v|\ll1$ (pair production rare, linear in pump power).
\end{frame}

\begin{frame}{Two-photon state (concept)}
	\begin{equation*}
		|\Psi\rangle = N\iint d^3k_s d^3k_i\,\Phi(k_s,k_i)\,\hat a^{\dagger}(k_s)\hat a^{\dagger}(k_i)|0\rangle + (\text{second pass}).
	\end{equation*}
	$\Phi$ encodes phase matching; interference requires overlap of idler modes from both passes.
\end{frame}

\section{Theory}
\begin{frame}{Signal rate and interference (schematic)}
	\begin{equation}
		R(\theta_s) = N\int d\omega_i\,d\theta_i\;[1 + t(\theta_i)\cos(\phi_0 + \tfrac{\omega_i}{c}\Delta l_i)]
	\end{equation}
	$t(\theta_i)$: object transmission; $\Delta l_i$: idler optical path difference. The cosine term yields oscillations used to sense optical thickness.
\end{frame}

\begin{frame}{Fit function used by authors}
	\begin{equation*}
		f(x) = y_0 + A\sin(\nu x + \phi)\,\exp\!\left[-\frac{(x-x_c)^2}{2\omega^2}\right].
	\end{equation*}
	The fitted center $x_c$ shifts when a plate (index $n$, thickness $d$) is inserted; shift maps to optical thickness $(n-1)d$.
\end{frame}

\section{Experimental setup}
\begin{frame}{Schematic (paper Fig.1)}
	\centering
	\includegraphics[width=0.85\linewidth]{fig-001.png}
	\\
	{\scriptsize Scheme and nomenclature for the theoretical analysis (paper Fig.1).}
\end{frame}

\begin{frame}{Full layout (paper Fig.2)}
	\centering
	\includegraphics[width=0.9\linewidth]{fig-000.png}
	\\
	{\scriptsize Experimental layout: pump (VBG), PPLN, ITO separation, parabolic mirror for idler, movable mirror Mi, VBG filters, sCMOS.}
	\\
	{\scriptsize Key params: 659.58 nm CW pump; 1-mm MgO:PPLN, poling 90\,\textmu m; sCMOS detection.}
\end{frame}

\section{Measurements}
\begin{frame}{Interference and FFT (paper Fig.3)}
	\centering
	\includegraphics[width=0.9\linewidth]{fig-002.png}
	\\
	{\scriptsize FFT of interference shows a peak near 1.26 THz; control (ITO) blocks interference.}
\end{frame}

\begin{frame}{Thickness sensing (paper Fig.4)}
	\centering
	\includegraphics[width=0.9\linewidth]{fig-003.png}
	\\
	{\scriptsize Plate insertion shifts envelope; fitted thickness agrees with caliper within a few percent.}
\end{frame}

\begin{frame}{Additional checks (paper Fig.5 \, Fig.6)}
	\centering
	\includegraphics[width=0.48\linewidth]{fig-004.png}\hfill
	\includegraphics[width=0.48\linewidth]{fig-005.png}
	\\
	{\scriptsize Left: pump-power linearity confirms low-gain SPDC. Right: idler angular distribution limits visibility.}
\end{frame}

\section{Discussion}
% Single clean Beamer presentation for the paper (fixed titlepage and TOC)
\documentclass{beamer}
\usetheme{Madrid}
\usecolortheme{seahorse}
\usepackage{amsmath,amssymb}
\usepackage{graphicx}
\usepackage{hyperref}
\graphicspath{{./}}

	itle[Terahertz quantum sensing]{Terahertz Quantum Sensing}
\subtitle{``Terahertz quantum sensing'' — M. Kutas et al., Sci. Adv. 2020}
\author[Your Name]{Prepared by: Your Name}
\date{August 26, 2025}

\begin{document}

% Title page
\begin{frame}[plain]
		itlepage
\end{frame}

% TOC
\begin{frame}{Outline}
		ableofcontents[hideallsubsections]
\end{frame}

\section{Motivation}
\begin{frame}{Why terahertz quantum sensing?}
	\begin{itemize}
		\item Terahertz (THz) carries material-specific contrast but detectors are inefficient.
		\item Use correlated photons: idler (THz) interacts with sample; signal (visible) is detected.
		\item Infer THz properties without direct THz detection.
	\end{itemize}
\end{frame}

\section{Key concepts}
\begin{frame}{Induced coherence without induced emission}
	\begin{itemize}
		\item SPDC produces signal (visible) + idler (THz) photon pairs.
		\item Two indistinguishable generation pathways interfere at the signal, dependent on idler path.
		\item Object transmissivity and idler phase modulate interference visibility.
	\end{itemize}
\end{frame}

\begin{frame}{Operator picture (intuition)}
	\begin{equation*}
		\hat a'_s = u\,\hat a_s + v\,\hat a_i^{\dagger},\qquad
		\hat a'_i = u\,\hat a_i + v\,\hat a_s^{\dagger},
	\end{equation*}
	with $|u|^2-|v|^2=1$. Low-gain: $|v|\\ll1$.
\end{frame}

\section{Theory}
\begin{frame}{Signal rate (schematic)}
	\begin{equation}
		R(\theta_s) = N\int d\omega_i\,d\theta_i\;\big[1 + t(\theta_i)\cos(\phi_0 + \tfrac{\omega_i}{c}\Delta l_i)\big]
	\end{equation}
\end{frame}

\section{Experiment}
\begin{frame}{Setup (Fig.1)}
	\centering
	\includegraphics[width=0.85\linewidth]{fig-001.png}
	\\
	{\scriptsize Schematic (paper Fig.1).}
\end{frame}

\begin{frame}{Layout (Fig.2)}
	\centering
	\includegraphics[width=0.9\linewidth]{fig-000.png}
	\\
	{\scriptsize Experimental layout and key components.}
\end{frame}

\section{Results}
\begin{frame}{Interference and FFT (Fig.3)}
	\centering
	\includegraphics[width=0.9\linewidth]{fig-002.png}
	\\
	{\scriptsize FFT peak near 1.26 THz.}
\end{frame}

\begin{frame}{Thickness sensing (Fig.4)}
	\centering
	\includegraphics[width=0.9\linewidth]{fig-003.png}
	\\
	{\scriptsize Fitted envelope shift vs PTFE thickness.}
\end{frame}

\begin{frame}{Checks (Fig.5 \& Fig.6)}
	\centering
	\includegraphics[width=0.48\linewidth]{fig-004.png}\hfill
	\includegraphics[width=0.48\linewidth]{fig-005.png}
	\\
	{\scriptsize Pump-power linearity and idler angular distribution.}
\end{frame}

\section{Conclusions}
\begin{frame}{Conclusions}
	\begin{itemize}
		\item Induced coherence enables THz sensing using visible detectors.
		\item Demonstrated thickness extraction and FFT signatures around 1.26 THz.
		\item Improvements: better idler collection, reduced loss, spectral control.
	\end{itemize}
\end{frame}

\begin{frame}{References}
	\footnotesize
	M. Kutas et al., "Terahertz quantum sensing," Sci. Adv. 6, eaaz8065 (2020).
\end{frame}

\end{document}
\end{frame}

\section{Measurements}
\begin{frame}{Interference and FFT (paper Fig.3)}
	\centering
	\includegraphics[width=0.9\linewidth]{fig-002.png}
	\\
	{\scriptsize FFT of interference shows peak at ~1.26 THz (collinear forward); inserting ITO in idler path removes interference (control).}
\end{frame}

\begin{frame}{Thickness sensing (paper Fig.4)}
	\centering
	\includegraphics[width=0.9\linewidth]{fig-003.png}
	\\
	{\scriptsize Inserting PTFE plates shifts the interference envelope; fitted thickness matches mechanical caliper to within a few percent.}
\end{frame}

\begin{frame}{Additional checks (paper Fig.5 and Fig.6)}
	\centering
	\includegraphics[width=0.48\linewidth]{fig-004.png}\hfill
	\includegraphics[width=0.48\linewidth]{fig-005.png}
	\\
	{\scriptsize Left: pump-power test (blocked/unblocked idler) shows linear behavior — low-gain regime. Right: idler angular distribution explains limited visibility.}
\end{frame}

\section{Discussion}
\begin{frame}{Interpretation and limitations}
	\begin{itemize}
		\item Interference due to indistinguishability of pair-creation events (quantum picture).
		\item Thermal idler occupation and up/down-conversion relevant for THz — included in theory.
		\item Practical limits: THz absorption, Fresnel losses, limited idler collection angle, asymmetry in measured traces.
	\end{itemize}
\end{frame}

\begin{frame}{How to present this paper}
	\begin{itemize}
		\item Start with problem statement (lack of THz detectors) and concept (ICWIE).
		\item Explain Fig.1 and Fig.2 clearly: where THz interacts and where visible is detected.
		\item Walk through Fig.3 (interference and FFT) and Fig.4 (thickness extraction).
		\item Finish with limitations and potential improvements.
	\end{itemize}
\end{frame}

\begin{frame}{References}
	\footnotesize
	M. Kutas et al., "Terahertz quantum sensing," Sci. Adv. 6, eaaz8065 (2020).\\
	Lemos et al., Nature 2014; Lahiri et al., PRA 2015; Chekhova \& Ou, Adv. Opt. Photon. 2016.
\end{frame}

\end{document}
\documentclass{beamer}
\usetheme{Madrid}
\usecolortheme{seahorse}
\usepackage{amsmath,amssymb}
\usepackage{graphicx}
\usepackage{hyperref}
\graphicspath{{./}}

	itle{Terahertz Quantum Sensing}
\subtitle{``Terahertz quantum sensing'' — M. Kutas et al., Sci. Adv. 2020}
\author{Prepared by: Your Name}
\date{August 26, 2025}

\begin{document}

\begin{frame}
		itlepage
\end{frame}

\begin{frame}{Outline}
		ableofcontents
\end{frame}

\section{Motivation}
\begin{frame}{Why terahertz quantum sensing?}
	\begin{itemize}
		\item Terahertz (THz) range carries useful contrast (thickness, molecular resonances) but lacks convenient detectors.
		\item Use correlated photons: idler in THz interacts with sample, signal (visible) is detected.
		\item Information transferred from THz to visible using induced coherence without induced emission (ICWIE).
	\end{itemize}
\end{frame}

\section{Key concepts}
\begin{document}

\begin{frame}
		itlepage
\end{frame}

\begin{frame}{Outline}
		ableofcontents
\end{frame}

\section{Motivation}
\begin{frame}{Why terahertz quantum sensing?}
	\begin{itemize}
		\item Terahertz (THz) range carries useful contrast (thickness, molecular resonances) but lacks convenient detectors.
		\item Use correlated photons: idler in THz interacts with sample, signal (visible) is detected.
		\item Information transferred from THz to visible using induced coherence without induced emission (ICWIE).
	\end{itemize}
\end{frame}

\section{Key concepts}
\begin{frame}{Induced coherence without induced emission}
	\begin{itemize}
		\item Two passes through the same nonlinear crystal generate two indistinguishable SPDC pathways.
		\item If idler modes are aligned and indistinguishable, signal photons interfere depending on idler path and object.
		\item This enables sensing/imaging without directly detecting THz photons.
	\end{itemize}
\end{frame}

\begin{frame}{SPDC operators (single-mode intuition)}
	\begin{equation*}
		\hat a'_s = u\,\hat a_s + v\,\hat a_i^{\dagger},\\qquad
		\hat a'_i = u\,\hat a_i + v\,\hat a_s^{\dagger},
	\end{equation*}
	with unitarity constraint $|u|^2-|v|^2=1$. Low-gain regime: $|v|\\ll1$.
\end{frame}

\begin{frame}{Two-photon state (schematic)}
	\begin{equation*}
		|\\Psi\\rangle = N_1\\iint d^3k_{s1}d^3k_{i1}\\,\\Phi(k_{s1},k_{i1})\\,\\hat a^{\\dagger}(k_{s1})\\hat a^{\\dagger}(k_{i1})|0\\rangle + (\\text{second pass}).
	\end{equation*}
	Phase-matching function $\\Phi$ enforces energy/momentum conservation; interference depends on alignment of idler modes and phases accumulated between passes.
\end{frame}

\section{Theory: signal rate}
\begin{frame}{Signal rate and interference (schematic)}
	\begin{equation}
		R(\\theta_s) = N\\!\\int d\\omega_i\\,d\\theta_i\\,\\ldots\\;\\big[1 + t(\\theta_i)\\cos(\\phi_0 + \\tfrac{\\omega_i}{c}\\Delta l_i)\\big]
	\end{equation}
	The cosine term carries the idler-path-dependent phase; varying idler optical length or inserting a sample (transmissivity $t$) shifts the interference envelope.
\end{frame}

  \tableofcontents
\end{frame}

\section{Motivation}
\begin{frame}{Why terahertz quantum sensing?}
  \begin{itemize}
    \item Terahertz (THz) range carries useful contrast (thickness, molecular resonances) but lacks convenient detectors.
    \item Use correlated photons: idler in THz interacts with sample, signal (visible) is detected.
    \item Information transferred from THz to visible using induced coherence without induced emission (ICWIE).
  \end{itemize}
\end{frame}

\section{Key concepts}
\begin{frame}{Induced coherence without induced emission}
  \begin{itemize}
    \item Two passes through the same nonlinear crystal generate two indistinguishable SPDC pathways.
    \item If idler modes are aligned and indistinguishable, signal photons interfere depending on idler path and object.
    \item This enables sensing/imaging without directly detecting THz photons.
  \end{itemize}
\end{frame}

\begin{frame}{SPDC operators (single-mode intuition)}
  \begin{equation*}
    \hat a'_s = u\,\hat a_s + v\,\hat a_i^{\dagger},\qquad
    \hat a'_i = u\,\hat a_i + v\,\hat a_s^{\dagger},
  \end{equation*}
  with unitarity constraint $|u|^2-|v|^2=1$. Low-gain regime: $|v|\ll1$.
\end{frame}

\begin{frame}{Two-photon state (schematic)}
  \begin{equation*}
    |\Psi\rangle = N_1\iint d^3k_{s1}d^3k_{i1}\,\Phi(k_{s1},k_{i1})\,\hat a^{\dagger}(k_{s1})\hat a^{\dagger}(k_{i1})|0\rangle + (\text{second pass}).
  \end{equation*}
  Phase-matching function $\Phi$ enforces energy/momentum conservation; interference depends on alignment of idler modes and phases accumulated between passes.
\end{frame}

\section{Theory: signal rate}
\begin{frame}{Signal rate and interference (schematic)}
  \begin{equation}
    R(\theta_s) = N\!\int d\omega_i\,d\theta_i\,\ldots\;\big[1 + t(\theta_i)\cos(\phi_0 + \tfrac{\omega_i}{c}\Delta l_i)\big]
  \end{equation}
  The cosine term carries the idler-path-dependent phase; varying idler optical length or inserting a sample (transmissivity $t$) shifts the interference envelope.
\end{frame}

\begin{frame}{Data analysis: fit used in paper}
  Authors fit the collinear signal envelope with:
  \begin{equation*}
    f(x) = y_0 + A\sin(\nu x + \phi)\,\exp\!\left[-\frac{(x-x_c)^2}{2\omega^2}\right].
  \end{equation*}
  Center $x_c$ maps to optical path; inserting a plate of refractive index $n$ and thickness $d$ shifts $x_c$ by an amount proportional to $(n-1)d$.
\end{frame}

\section{Experimental setup}
\begin{frame}{Schematic (Fig.1)}
  \begin{figure}
    \centering
    \includegraphics[width=0.85\linewidth]{fig-001.png}
    \caption{Scheme and nomenclature for the theoretical analysis (paper Fig.1).}
  \end{figure}
  Notes: single-crystal Michelson-like interferometer; idler (THz) interacts with object O in free space between passages.
\end{frame}

\begin{frame}{Full layout (Fig.2)}
  \begin{figure}
    \centering
    \includegraphics[width=0.9\linewidth]{fig-000.png}
    \caption{Experimental layout (paper Fig.2): pump (VBG), PPLN, ITO separation, parabolic mirror, moveable mirror Mi, VBG filters, sCMOS.}
  \end{figure}
  Key parameters: pump 659.58 nm CW laser; 1-mm MgO:PPLN, poling period 90 \textmu m; sCMOS detection after VBG filters and grating.
\end{frame}

\section{Measurements}
\begin{frame}{Interference and FFT (Fig.3)}
  \begin{figure}
    \centering
    \includegraphics[width=0.9\linewidth]{fig-002.png}
    \caption{Interference in collinear forward spot (Stokes/anti-Stokes) and FFT peaks (~1.26 THz).}
  \end{figure}
  Control: placing ITO in idler path removes interference peak (no THz reaches object).
\end{frame}

\begin{frame}{Thickness sensing (Fig.4)}
  \begin{figure}
    \centering
    \includegraphics[width=0.9\linewidth]{fig-003.png}
    \caption{Envelope shift vs PTFE plate thickness; fitted values compared to caliper.}
  \end{figure}
  Result: thickness extraction accurate to a few percent; demonstrated at various plate thicknesses (0.5–5 mm).
\end{frame}

\begin{frame}{Additional checks (Fig.5, Fig.6)}
  \begin{figure}
    \centering
    \includegraphics[width=0.48\linewidth]{fig-004.png}\hfill
    \includegraphics[width=0.48\linewidth]{fig-005.png}
    \caption{Left: pump-power test (blocked/unblocked idler). Right: idler angular distribution for various pump waists.}
  \end{figure}
  These show low-gain SPDC (linear pump dependence) and that limited idler angular collection reduces visibility.
\end{frame}

\section{Discussion}
\begin{frame}{Interpretation, limitations, outlook}
  \begin{itemize}
    \item Quantum explanation: indistinguishability of generation paths leads to interference in detected signal.
    \item Practical limits: THz absorption in PPLN, Fresnel losses, limited coupling angles, thermal idler occupation.
    \item Outlook: improve collection efficiency, reduce loss, extend to spectroscopic contrast and imaging.
  \end{itemize}
\end{frame}

\begin{frame}{How to present this paper}
  \begin{itemize}
    \item Start with problem statement (lack of THz detectors), explain the ICWIE idea with Fig.1.
    \    \item Walk through Fig.2 (setup) step-by-step.
    \item Show experimental data (Fig.3–Fig.5) and explain fit-to-thickness mapping.
    \item Finish with limitations and potential next steps.
  \end{itemize}
\end{frame}

\begin{frame}{References}
  \footnotesize
  M. Kutas et al., "Terahertz quantum sensing," Sci. Adv. 6, eaaz8065 (2020).\\
  Lemos et al., Nature 2014; Lahiri et al., PRA 2015; Chekhova & Ou, Adv. Opt. Photon. 2016.
\end{frame}

\end{document}

\documentclass{beamer}
\usetheme{Madrid}
\title{Ultrafast and Reversible Control of the Exchange Interaction in Mott Insulators}
\author{A. Singer, S. Parameswaran, et al.}
\date{August 26, 2025}

\begin{document}

\begin{frame}
		\titlepage
\end{frame}

\begin{frame}{Outline}
	\tableofcontents
\end{frame}

\section{Concepts}
\documentclass{beamer}
\usetheme{Madrid}
\usepackage{amsmath,amssymb}
\usepackage{graphicx}
\usepackage{hyperref}
\graphicspath{{./}}
	itle{Terahertz Quantum Sensing}
\subtitle{"Terahertz quantum sensing" — M. Kutas, B. Haase, P. Bickert, F. Riexinger, D. Molter, G. von Freymann (Sci. Adv. 2020)}
\author{Mirco Kutas, Björn Haase, et al. \\ Prepared by: Your Name}
\date{August 26, 2025}

\begin{document}

\begin{frame}
		itlepage
\end{frame}

\begin{frame}{Outline}
		ableofcontents
\end{frame}

\section{Motivation}
\begin{frame}{Why terahertz quantum sensing?}
	\begin{itemize}
		\item Terahertz (THz) region contains useful material information but lacks widely available, sensitive detectors.
		\item Quantum sensing with undetected photons transfers sample information from THz to visible wavelengths where detectors are efficient.
		\item Goal of paper: demonstrate free-space THz interaction with a sample and read out thickness information using visible photons.
	\end{itemize}
\end{frame}

\section{Key concepts}
\begin{frame}{Induced coherence without induced emission (ICWIE)}
	\begin{itemize}
		\item Photon pairs (signal: visible, idler: THz) are generated by SPDC in a nonlinear crystal.
		\item Two indistinguishable generation pathways (first and second pass) produce interference in the detected signal even when the idler is not detected.
		\item This enables sensing in spectral ranges where direct detection is hard.
	\end{itemize}
\end{frame}

\begin{frame}{SPDC and Bogoliubov transforms}
	In each passage through the crystal the input/output field operators are related by Bogoliubov transformations (single-mode notation):
	\begin{equation*}
		\hat{a}'_s = u\,\hat{a}_s + v\,\hat{a}_i^{\dagger},\qquad
		\hat{a}'_i = u\,\hat{a}_i + v\,\hat{a}_s^{\dagger}.
	\end{equation*}
	Unitarity implies $|u|^2-|v|^2=1$. In the low-gain regime $|v|\ll1$ and pair creation probability is small.
\end{frame}

\begin{frame}{Two-photon state (multimode) — intuition}
	The produced two-photon state is a superposition of pairs created on the first and the second pass:
	\begin{equation*}
		|\Psi\rangle = N_1\int d^3k_{s1}d^3k_{i1}\,\Phi(k_{s1},k_{i1})\,\hat a^{\dagger}(k_{s1})\hat a^{\dagger}(k_{i1})|0\rangle + N_2(\text{second pass}).
	\end{equation*}
	Phase matching (encoded in $\Phi$) and the alignment of idler modes determine interference visibility.
\end{frame}

\section{Theory: signal rate and interference}
\begin{frame}{Signal rate — origin of interference}
	The detector signal rate can be written schematically as:
	\begin{equation}
		R(\theta_s) = N\int d\omega_i\,d\theta_i\,\ldots\;\big[1 + t(\theta_i)\cos(\phi_0 + \tfrac{\omega_i}{c}\Delta l_i)\big],
	\end{equation}
	where $t(\theta_i)$ is object transmission, $\Delta l_i$ is idler optical path difference, and $\phi_0$ summarizes constant phases. Interference arises from the cosine term.
\end{frame}

\begin{frame}{Interference fit used to extract shifts}
	The authors fit the collinear signal envelope with:
	\begin{equation*}
		f(x) = y_0 + A\sin(\nu x + \phi)\,\exp\!\left[-\frac{(x-x_c)^2}{2\omega^2}\right],
	\end{equation*}
	The fitted center $x_c$ shifts when a plate of refractive index $n$ and thickness $d$ is inserted; the shift maps to optical thickness $n d$.
\end{frame}

\section{Experimental setup}
\begin{frame}{Setup (schematic) — Fig.1}
	\begin{figure}
		\centering
		\includegraphics[width=0.9\linewidth]{fig-001.png}
		\caption{Scheme and nomenclature for the theoretical analysis (from paper). Signal (s) and idler (i) modes pass through PPLN twice; idler interacts with object O in free space.}
	\end{figure}
	Explanation:
	\begin{itemize}
		\item Single-crystal Michelson-like interferometer: pump and signal reflected back for second pass.
		\item Idler path includes object O; its optical length and transmissivity modify interference.
	\end{itemize}
\end{frame}

\begin{frame}{Full experimental layout — Fig.2}
	\begin{figure}
		\centering
		\includegraphics[width=0.9\linewidth]{fig-000.png}
		\caption{Experimental layout: pump (VBG), PPLN, ITO separation, parabolic mirror for idler, moveable mirror Mi, VBG filters and sCMOS detection.}
	\end{figure}
	Notes:
	\begin{itemize}
		\item Pump: 659.58 nm CW laser (narrow linewidth), focused into 1-mm PPLN.
		\item ITO glass transmits idler (THz) and separates visible signal for detection.
	\end{itemize}
\end{frame}

\begin{frame}{Observed interference (Fig.3) — explanation}
	\begin{figure}
		\centering
		\includegraphics[width=0.9\linewidth]{fig-002.png}
		\caption{Interference in collinear forward spot (Stokes and anti-Stokes) and corresponding FFT peaks at ~1.26 THz.}
	\end{figure}
	Key points:
	\begin{itemize}
		\item FFT peak frequency matches phase-matching expectation (collinear forward: ~1.26 THz).
		\item Inserting ITO in idler path removes interference (control), confirming idler-mediated effect.
	\end{itemize}
\end{frame}

\begin{frame}{Quantum sensing demonstration (Fig.4)}
	\begin{figure}
		\centering
		\includegraphics[width=0.9\linewidth]{fig-003.png}
		\caption{Shift of interference envelope when PTFE plates of different thicknesses are inserted; fitted thickness vs caliper measurement.}
	\end{figure}
	Result: thickness extracted from interference agrees with mechanical measurement within a few percent.
\end{frame}

\begin{frame}{Checks and additional data (Fig.5, Fig.6)}
	\begin{figure}
		\centering
		\includegraphics[width=0.48\linewidth]{fig-004.png}\hfill
		\includegraphics[width=0.48\linewidth]{fig-005.png}
		\caption{Left: pump-power test (blocked/unblocked idler) showing linear dependence (low-gain). Right: idler angular distribution for various pump waists.}
	\end{figure}
	These support the low-gain SPDC regime and explain visibility limits due to limited idler angular collection.
\end{frame}

\section{Discussion and limitations}
\begin{frame}{Interpretation and limitations}
	\begin{itemize}
		\item Quantum explanation: indistinguishability of creation events causes interference even with undetected idler.
		\item Thermal idler occupation and up/down-conversion processes are relevant in THz; authors include thermal terms in theory.
		\item Practical limitations: THz absorption in PPLN, Fresnel losses, limited idler collection angle, and asymmetry in measured interference.
	\end{itemize}
\end{frame}

\section{Presentation tips}
\begin{frame}{How to present this to your professor}
	\begin{itemize}
		\item Use Fig.1 and Fig.2 to explain the concept and the apparatus step-by-step.
		\item Show Fig.3 (interference & FFT) to connect theory and measurement.
		\item Explain the fit (show equation) and how shifts map to thickness (Fig.4).
		\item End with limitations and potential improvements.
	\end{itemize}
\end{frame}

\begin{frame}{References}
	\footnotesize
	M. Kutas et al., "Terahertz quantum sensing," Sci. Adv. 6, eaaz8065 (2020).\\
	Further reading: Lemos et al., Nature 2014; Lahiri et al., PRA 2015; Chekhova & Ou, Adv. Opt. Photon. 2016.
\end{frame}

\end{document}
		\section{Experimental setup}
		\begin{frame}{Main elements}
			\begin{itemize}
				\item Pump: continuous-wave, frequency-doubled laser at 659.58 nm (narrow linewidth).
				\item Nonlinear medium: 1 mm MgO-doped periodically poled LiNbO3 (PPLN), poling period $\Lambda=90\,\mu$m.
				\item Separation: ITO-coated glass separates visible (signal) and THz (idler) radiation.
				\item Idler path: parabolic mirror to collimate THz, mirror on piezo/fine stage provides variable path length.
				\item Detection: signal filtered by VBGs and a transmission grating, then recorded on sCMOS camera (visible detection).
			\end{itemize}
			% TODO: add figure from the paper (Fig.2) here when available
		\end{frame}

		\begin{frame}{Operating regime and practical points}
			\begin{itemize}
				\item Low-gain SPDC (V0 \ll 1): signal intensity linear in pump power.
				\item Thermal idler occupation matters in THz: $N_{th}=1/(e^{\hbar\omega_i/k_BT}-1)$ appears in theory.
				\item Only idler angles up to a maximum ($\theta_{\max}\approx5^\circ$ inside crystal) can be coupled back; limits visibility.
				\item Losses (absorption, Fresnel reflections) reduce visibility compared to ideal theory.
			\end{itemize}
		\end{frame}

		\section{Measurements and results}
		\begin{frame}{Interference and FFT peaks}
			\begin{itemize}
				\item Observed interference in both Stokes (down-conversion) and anti-Stokes (up-conversion) regions.
				\item Fast Fourier Transform of interference envelope shows peaks at ~1.26 THz (collinear forward) and ~0.47 THz (collinear backward) — matches phase matching.
				\item Inserting an ITO glass in idler path blocks THz; interference disappears (control test).
			\end{itemize}
			% placeholder for Fig.3 (interference and FFT)
		\end{frame}

		\begin{frame}{Terahertz quantum sensing (thickness measurement)}
			\begin{itemize}
				\item PTFE plates of known thickness inserted in idler path cause a shift of the interference envelope center $x_c$.
				\item Using measured refractive index $n_{\mathrm{PTFE}}=1.42\pm0.01$ and the shift, the optical thickness (physical thickness) is extracted.
				\item Results agree with caliper measurements within a few percent (example: 2.9% accuracy for 5 mm plate in Stokes).
			\end{itemize}
		\end{frame}

		\section{Discussion}
		\begin{frame}{Interpretation and robustness}
			\begin{itemize}
				\item Interference arises from indistinguishability of pair creation events (quantum explanation). Classical explanations exist but do not capture all observations.
				\item Tests: blocking idler, varying pump power (linear behavior consistent with low-gain), adding ITO — confirm quantum sensing mechanism.
				\item Limitations: asymmetry in measured interference, THz absorption in crystal, limited angular collection.
			\end{itemize}
		\end{frame}

		\begin{frame}{Applications and outlook}
			\begin{itemize}
				\item Noninvasive thickness measurements and imaging in THz range without THz detectors.
				\item Potential for industrial non-destructive testing and quantum-enhanced metrology.
				\item Improvements: increase collection efficiency, reduce losses, extend to spectroscopic contrast.
			\end{itemize}
		\end{frame}

		\section{Practical notes for your presentation}
		\begin{frame}{How to present this paper to your professor}
			\begin{itemize}
				\item Start with the problem: lack of good THz detectors and the idea of using undetected photons.
				\item Explain the experiment schematic clearly (use Fig.2 from the paper).
				\item Walk through the theory intuitively: SPDC, indistinguishability, how path changes give interference.
				\item Show data (interference and FFT) and explain how thickness is extracted (fit and refractive index).
				\item End with limitations and future directions — be ready to discuss thermal idler contribution and low-gain regime.
			\end{itemize}
		\end{frame}

		\begin{frame}{Further reading}
			\begin{itemize}
				\item Lemos et al., Nature 2014 — Quantum imaging with undetected photons.
				\item Lahiri et al., Phys. Rev. A 2015 — Theory of quantum imaging with undetected photons.
				\item Reviews on SPDC and nonlinear interferometers: Chekhova & Ou, Adv. Opt. Photon. 2016.
				\item Practical: Overleaf Beamer guide for slide polish.
			\end{itemize}
		\end{frame}

		\begin{frame}{References}
			\footnotesize
			M. Kutas et al., "Terahertz quantum sensing," Sci. Adv. 6, eaaz8065 (2020).\\
			See paper for full author list and experimental details.
		\end{frame}

		\end{document}
